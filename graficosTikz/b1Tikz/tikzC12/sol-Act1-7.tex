\begin{tikzpicture}[scale=0.5]
    \draw[help lines, opacity=0.4] (0,-1) grid (10,9); %cuadricula

    \draw[thick] (1,1) -- (7,1) -- (7,7) -- (1,7) --cycle; %dibujo cuadrilatero

    %Etiquetas vertices y puntos medios
    \node[above left] at (1,1) {A};
    \coordinate (A) at (1,1);
    \node[above right] at (7,1) {B};
    \coordinate (B) at (7,1);
    \node[above right] at (7,7) {C};
    \coordinate (C) at (7,7);
    \node[above left] at (1,7) {D};
    \coordinate (D) at (1,7);
    \node[above left] at (1,4) {E};
    \coordinate (E) at (1,4);
    \node[below] at (4,1) {F};
    \node[above right] at (7,4) {G};
    \node[above] at (4,7) {H};

    %segmentos de los vertices a los puntos medios
    \draw[thick,green, name path=AH] (1,1) -- (4,7); %AH
    \draw[thick,green, name path=CF] (7,7) -- (4,1); %CF
    \draw[thick,green, name path=BE] (7,1) -- (1,4); %BE
    \draw[thick,green, name path=DG] (1,7) -- (7,4); %DG

    %nombrando las intersecciones
    \path[name intersections={of=DG and AH, by={P}}];
    \path[name intersections={of=DG and CF, by={Q}}];
    \path[name intersections={of=AH and BE, by={R}}];
    \path[name intersections={of=BE and CF, by={S}}];

    %coloreando de azul el cuadrado chico
    \draw[fill=cyan, ultra thick, opacity=0.3] (P) -- (Q) -- (S) -- (R) --cycle;

    %coordenadas de los otros cuadrados
    \coordinate (I) at (4.6,8.2);
    \coordinate (K) at (9.4,5.8);
    \coordinate (L) at (8.2,3.4);
    \coordinate (J) at (3.4,-0.2);

    %coloreando de azul los demas cuadrados
    \draw[fill=cyan, thick, opacity=0.3] (I) -- (C) -- (Q) -- (P) --cycle;
    \draw[fill=cyan, thick, opacity=0.3] (C) -- (K) -- (L) -- (Q) --cycle;
    \draw[fill=cyan, thick, opacity=0.3] (Q) -- (L) -- (B) -- (S) --cycle;
    \draw[fill=cyan, thick, opacity=0.3] (R) -- (S) -- (J) -- (A) --cycle;

    %muestra de construccion de los cuadrados
    \draw[red,thin] (A) -- (R) -- (E) --cycle;
    \draw[blue,thick] (E) -- (R) -- (P) -- (D) --cycle;

\end{tikzpicture}