\chapter{Clase 9. PROBLEMAS}
\textbf{04/03/2025}

Ejercicios p.14 del problemario.
\begin{excercise}
    1.13 «Pagué doce centavos por los huevos que compré al tendero», explicó la cocinera, «pero le hice darme dos huevos extra porque eran muy pequeños. Eso hizo que el total sumara un centavo menos por docena que el primer precio que me dio.» ¿Cuántos huevos compró la cocinera?
\end{excercise}

\begin{excercise}
    1.14 La señora Wiggs le explicaba a Lovey Mary que ahora tiene una plantación cuadrada de repollos más grande que la del año pasado, y que por lo tanto tendrá 211 repollos más. ¿Cuántos repollos tendrá este año la señora Wiggs.
\end{excercise}

\begin{excercise}
    1.15 Tras recoger 770 castañas, tres niñas las dividieron de modo que las cantidades recibidas guardaran la misma proporción que sus edades. Cada vez que Mary se quedaba con cuatro castañas, Nellie tomaba tres, y por cada seis que recibía Mary, Susie tomaba siete. ¿Cuántas castañas recibió cada niña?
\end{excercise}
