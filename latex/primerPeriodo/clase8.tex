\chapter{Clase 8. Problemas de aritméticos y geométricos}
\textbf{27/02/2025}

%para que concuerde con el número del problemario de pensamiento

\begin{excercise}
    Un tren directo sale de Moscú hacia Leningrado a 90 km por hora. Otro tren directo sale de Leningrado a Moscú a 60 km por hora. ¿A qué distancia están los trenes entre sí media hora antes de que se crucen?
\end{excercise}
\textbf{SOLUCIÓN}
Sea $T_{ML}$ el tren que va de Moscú a Leningrado y $T_{LM}$ el tren que va de Leningrado a Moscú.
Como el problema indica que los dos trenes se cruzan en determinado momento, entonces media hora antes $T_{ML}$ estaba a 45 km del punto en el cual se iba a encontrar con $T_{LM}$, y $T_{LM}$ media hora antes del punto de encuentro con el otro tren estaba 30 km, por lo cual al sumar estas dos distancias, los trenes estaban a 75 km antes de cruzarse. 

\begin{excercise}
    Dos ciclistas empiezan juntos un recorrido de entrenamiento, uno empezando desde Moscú, el otro desde Simferopol. Cuando los ciclistas están separados por 180 millas, una mosca entra en escena. Empezando desde el hombro de uno de los ciclistas, la mosca vuela hasta el otro ciclista. Al llegar al otro, da la vuelta y vuelve. La mosca incansable continúa yendo de uno a otro hasta que los dos se encuentran; entonces se posa sobre la nariz de uno de los ciclistas.
    La velocidad de la mosca es de 30 millas por hora. La velocidad de los ciclistas es de 15 millas por hora. ¿Cuántas millas ha recorrido la mosca?
\end{excercise}
\textbf{SOLUCIÓN}

Cómo cada ciclista individualmente recorre 15 millas por hora, entonces con cada hora la distancia que hay entre ellos dos se reduce en 30 millas. Utilizando la fórmula de velocidad nos da que el tiempo que tardan en encontrarse es de seis horas y como la mosca es capaz de recorrer 30 millas por hora, al multiplicar la cantidad de tiempo que los ciclistas tarden en encontrarse por la velocidad de la mosca nos da que recorre 180 millas, hasta el momento en que los ciclistas se encuentran.

\begin{excercise}
    En 855, el emperador de China Yang Suen tenía que cubrir un puesto importante al que aspiraban dos mandarines de títulos equivalentes. El emperador decidió elegir a aquel que resolviera primero el siguiente problema:
    El jefe de unos bandidos decía a sus hombres: «Hemos robado unas piezas de tela... Si cada uno de nosotros toma seis, quedarán cinco piezas. Pero si cada uno de nosotros quiere siete, nos faltarán ocho.» ¿Cuántos eran los ladrones?
    Como en aquella época no se conocía el álgebra en la China, los mandarines tuvieron que resolver el problema valiéndose de la aritmética. Nosotros nos proponemos hacer lo mismo.
\end{excercise}
\textbf{SOLUCIÓN}

El problema indica que cuando cada ladrón agarra seis pedazos de tela sobran cinco, y al momento de hacer una repartición con un pedazo de tela más, los primeros agarran, los cinco pedazos que sobraron entonces cinco, ya tienen siete pedazos de tela pero hacen falta ocho para que cada uno tenga siete pedazos de tela, por lo cual, si sumamos los 5 pedazos que sobran con los 8 que faltan podemos concluir que son 13 ladrones. 

\begin{excercise}
    Dos coches salen al mismo tiempo del punto A para cumplir una carrera de regularidad. El circuito tiene más de un kilómetro de longitud. Durante toda la competición, en la que darán varias vueltas al circuito, cada coche mantendrá fija su velocidad. Como un coche va más rápido que el otro, en determinados momentos los coches se cruzarán. El primer cruce se produce a 150 metros del punto A. ¿A qué distancia del punto A se cruzarán por segunda vez?
\end{excercise}
\textbf{SOLUCIÓN}
Los coches se volverán a cruzar a 300 m del punto A

\begin{excercise}
    «El café se me enfría » dijo Óscar, «y sigo esperando por la crema y el azúcar.»
    La joven no parecía preocupada.
    «Lo siento, señor » dijo ella. «Voy a traérselos especialmente, ya que la mayoría de la gente nunca los pide aquí.»
    «¿Conque no, eh? Estuve observando a los otros clientes » dijo Óscar « y noté que doce se sirvieron azúcar, siete se sirvieron crema, tres se sirvieron ambas cosas, y únicamente dos no se sirvieron nada.»
    ¿Cuántos clientes además de Óscar había allí?
\end{excercise}
\textbf{SOLUCIÓN}
Con diagramas de ven, primeramente se marcan las intersecciones, es decir, los tres que se sirvieron ambas cosas, y de ahí a la cantidad total de personas que sirvieron crema,, se le resta la intersección, y de igual modo con los que se sirvieron azúcar. Sumando cada cantidad obtenida más las dos personas que no se sirvieron nada, y Oscar nos da un total de 19 personas.
\begin{gather*}
    A : \text{Personas que se sirvieron azúcar} \;
    C : \text{Personas que se sirvieron crema}
\end{gather*}

\begin{center}
    \begin{venndiagram2sets}[labelB = {C}, labelOnlyA = {4}, labelAB = {3}, labelOnlyB = {9}, labelNotAB = {2}]
    \end{venndiagram2sets}
\end{center}


\begin{excercise}
    En la multiplicación que se muestra en la figura, ciertos dígitos han sido reemplazados por un *.
    Encuentra los dígitos que han sido reemplazados por *.
\end{excercise}

\begin{figure*}
    \centering
    \begin{tabular}{c c c c c c | c}
        &&&*&1&*&  1 \\
        &&$\times$&&&&  2 \\
        &&&3&*&2&  3 \\
        \hline
        &&&*&3&*&  4\\
        &3&*&2&*&&  5 \\
        *&2&*&5&&&  6\\
        \hline
        1&*&8&*&3&0&  7\\
        A&B&C&D&E&F \\
    \end{tabular}    
\end{figure*}


\textbf{SOLUCIÓN}

    \begin{center}    
        \begin{tabular}{c c c c c c | c}
            &&&4&1&5&  1 \\
            &&$\times$&&&&  2 \\
            &&&3&8&2&  3 \\
            \hline
            &&&8&3&0&  4\\
            &3&3&2&0&&  5 \\
            1&2&4&5&&&  6\\
            \hline
            1&5&8&5&3&0&  7\\
            A&B&C&D&E&F \\
        \end{tabular}
    \end{center}

Para esta explicación especificaremos el número o asterisco al que nos referimos poniendo por delante su posición entre paréntesis.
El primer paso para hallar el valor de *(F1) fue ver que número del 1-9 multiplicado por 2(F3) podría darme un cero en la posición de las unidades para que concordase con 0(F7), en este caso fue el 5.
Para *(E3) era ver que número multiplicado por 5 tenía una unidad que al ser sumado con la decena de la fila 3 me diera el 3(E7), en este caso habían dos posibilidades, el 4 y el 8. Escogí el 8, pues cumplía otra condición necesaria la cual es que multiplicando 8(E3) $\times$ 1(E3) y sumando la cifra de las decenas de 8(E3)$\times$5(F1) debía terminar en dos. Del mismo modo procedí con los números restantes, observando que números multiplicados por otros tenían cifras que ya conocía en la posición de las unidades.
\\

\begin{excercise}
    Martín estaba tendiendo un alambrado en el fondo del terreno.
    « Me dijiste que pensabas poner los postes con una separación de un metro entre uno y otro» comentó Lucrecia, « pero ahora veo que están más separados».
    «No creí que lo notaras »dijo Martín. « Resulta que me faltaron cuatro postes y tuve que ponerlos a metro y medio uno del otro.»
    ¿Cuánto mide el largo del alambrado?
\end{excercise}

\textbf{SOLUCIÓN}
En la siguiente figura $P$ representa el último poste que se colocó mientras que $Q$ es el final del alambrado y $P_1, P_2, P_3, P_4$ son los postes faltantes

\begin{figure*}[ht!]
    \begin{tikzpicture}
        \draw[<->] (-8,0) -- (6,0);
        %Cada punto representa un poste
        \fill (-6,0) circle (1pt);
        \node[above] at (-6,0) {\footnotesize $P$ };
        
        \fill (-4,0) circle (1pt);
        \node[above] at (-4,0) {\footnotesize $P_1$};
        
        \fill (-2,0) circle (1pt);
        \node[above] at (-2,0) {\footnotesize $P_2$};
        
        \fill (0,0)  circle (1pt);
        \node[above] at (0,0) {\footnotesize $P_3$};
        
        \fill (2,0)  circle (1pt);
        \node[above] at (2,0) {\footnotesize $P_4$};
        
        \fill (4,0)  circle (1pt);
        \node[above] at (4,0) {\footnotesize $Q$};

        %rectas que marcan la distancia entre cada punto
        \draw[<->] (-6,-0.5) -- (-4,-0.5);
        \node at (-5,-1) {\textbf{$1m$}};
        
        \draw[<->] (-4,-0.5) -- (-2,-0.5);
        \node at (-3,-1) {\textbf{$1m$}};
        
        \draw[<->] (-2,-0.5) -- (0,-0.5);
        \node at (-1,-1) {\textbf{$1m$}};
        
        \draw[<->] (0,-0.5) -- (2,-0.5);
        \node at (1,-1) {\textbf{$1m$}};
        
        \draw[<->] (2,-0.5) -- (4,-0.5);
        \node at (3,-1) {\textbf{$1m$}};
    \end{tikzpicture}
\end{figure*}

Notemos que si tenemos 6 postes, la distancia que cubren es de $5m$, de aquí obtenemos la primera ecuación

\begin{gather*}
    l = n-1 \qquad \text{Donde $l:=$ la longitud del alambrado y $n:=$ la cantidad de postes}
\end{gather*}

Pero esto es solo cuando tenemos la cantidad suficiente de postes, como hicieron falta cuatro postes entonces $l = (n-4)-1$ y como la distancia ahora es de $1.5m$ entre cada poste tenemos

\begin{gather*}
    l = \frac{3}{2}(n-5) \\
    \text{Como la longitud del alambrado no cambia entonces podemos igualar ambas ecuaciones}\\
    n-1 = \frac{3}{2}(n-5) \\
    2n-2 = 3n - 15\\
    3n - 2n = 15 -2\\
    n = 13
\end{gather*}

Sustituyendo tenemos $L = 13-1 = 12$, por lo tanto la longitud del alambrado es de $12m$.