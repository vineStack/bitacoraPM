%!TEX root = main.tex

\usepackage[a4paper,margin=1in]{geometry}

\usepackage{titlesec}
% Redefinir el formato del capítulo
\titleformat{\chapter}[block]
  {\normalfont\huge\bfseries} % Fuente del título (normal, tamaño, en negrita)
  {} % No mostrar el número de capítulo
  {0pt} % Espacio entre el número y el título
  {} % El formato del título, vacío para no poner "Capítulo"
\titlespacing*{\chapter}{0pt}{-10pt}{20pt} % Ajusta el espacio antes y después del título

%\usepackage[spanish]{babel}
\usepackage[spanish,es-noshorthands]{babel}
\usepackage[utf8]{inputenc}
\usepackage[T1]{fontenc}
\usepackage{amssymb}
\usepackage{amsmath}
\usepackage{amsfonts}

\usepackage{graphicx}
\graphicspath{{../images/}} %Este wachin ayuda accededer a poder ingresar a ella desde lugares con distintas jerarquias

\usepackage{hyperref}
\usepackage{caption}
\usepackage{tikz}
\usepackage{wrapfig}
\usepackage{float}
\usepackage{venndiagram}
\usepackage{textcomp}%para agregar el simbolo de centavo

\newtheorem{ejem}{Ejemplo}[section]
\newtheorem{excercise}{Ejercicio}
