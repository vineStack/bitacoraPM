\section{Clase 14.}
\textbf{11/03/2025}

1.6 Los hexágonos que se muestran en la figura (1.16), tienen idéntico tamaño y han sido sombreadas parte de su áreas.22 Intuición lógica y pensamiento lateral Ordena a los hexágonos de acuerdo a la cantidad de área sombreada. Figura 1.16

1.7 En la figura (1.17) se muestran dos rectángulos. ¿cuál de los rectángulos tiene mayor área?

1.8 Esta orquesta comprende 180 ejecutantes de los cuales:
\begin{enumerate}
    \item 6 no tocan más que violonchelo.
    \item 24 tocan violonchelo y violín, pero no viola.
    \item 12 tocan violonchelo y viola, pero no violín.
    \item 6 tocan violín y viola, pero no violonchelo.
    \item Además, sabemos que 63 músicos tocan el violín, 54 el violonchelo y 36 la viola.
\end{enumerate}
¿Cuántos ejecutantes no tocan ninguno de estos instrumentos?

\begin{minipage}[t]{0.6\textwidth}
    \noindent
    1.9 Cada letra representa a una cifra distinta. Hallar el valor, cada letra; la letra R debe ser distinta de cero.
\end{minipage}
\begin{minipage}
    \begin{tabular}{ccccc}
        &R&A&M&A\\
        +&R&A&M&A\\
        &R&A&M&A\\\hline
        A&R&B&O&L
    \end{tabular}
\end{minipage}