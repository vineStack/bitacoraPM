\section{Clase 12. Resolución de la sección 1.5 del problemario}
\textbf{11/03/2025}

\textbf{Actividad 1.7}
En un cuadrado se localizan los puntos medios de sus lados y se trazan segmentos de sus vértices a los puntos medios como se muestra en la figura
¿ Cuánto vale él área del cuadrado A, B,C, D entre el área
del cuadrado azul?\bigskip
\par
\begin{minipage}[c]{0.48\textwidth}
    \centering
    \begin{tikzpicture}[scale=0.5]
    \draw[help lines, opacity=0.4] (0,-1) grid (10,9); %cuadricula

    \draw[thick] (1,1) -- (7,1) -- (7,7) -- (1,7) --cycle; %dibujo cuadrilatero

    %Etiquetas vertices y puntos medios
    \node[above left] at (1,1) {\small A};
    \node[above right] at (7,1) {\small B};
    \node[above right] at (7,7) {\small C};
    \node[above left] at (1,7) {\small D};
    \node[above left] at (1,4) {\small E};
    \node[below] at (4,1) {\small F};
    \node[above right] at (7,4) {\small G};
    \node[above] at (4,7) {\small H};

    %segmentos de los vertices a los puntos medios
    \draw[thick,green, name path=AH] (1,1) -- (4,7); %AH
    \draw[thick,green, name path=CF] (7,7) -- (4,1); %CF
    \draw[thick,green, name path=BE] (7,1) -- (1,4); %BE
    \draw[thick,green, name path=DG] (1,7) -- (7,4); %DG

    %nombrando las intersecciones
    \path[name intersections={of=DG and AH, by={P}}];
    \path[name intersections={of=DG and CF, by={Q}}];
    \path[name intersections={of=AH and BE, by={R}}];
    \path[name intersections={of=BE and CF, by={S}}];

    %coloreando de azul el cuadrado chico
    \draw[fill=cyan, ultra thick,cyan,opacity=0.3] (P) -- (Q) -- (S) -- (R) --cycle;

\end{tikzpicture}
    \captionof{figure}{Problema}
\end{minipage}
\begin{minipage}[c]{0.48\textwidth}
    \centering
    \begin{tikzpicture}[scale=0.5]
    \draw[help lines, opacity=0.4] (0,-1) grid (10,9); %cuadricula

    \draw[thick] (1,1) -- (7,1) -- (7,7) -- (1,7) --cycle; %dibujo cuadrilatero

    %Etiquetas vertices y puntos medios
    \node[above left] at (1,1) {A};
    \coordinate (A) at (1,1);
    \node[above right] at (7,1) {B};
    \coordinate (B) at (7,1);
    \node[above right] at (7,7) {C};
    \coordinate (C) at (7,7);
    \node[above left] at (1,7) {D};
    \coordinate (D) at (1,7);
    \node[above left] at (1,4) {E};
    \coordinate (E) at (1,4);
    \node[below] at (4,1) {F};
    \node[above right] at (7,4) {G};
    \node[above] at (4,7) {H};

    %segmentos de los vertices a los puntos medios
    \draw[thick,green, name path=AH] (1,1) -- (4,7); %AH
    \draw[thick,green, name path=CF] (7,7) -- (4,1); %CF
    \draw[thick,green, name path=BE] (7,1) -- (1,4); %BE
    \draw[thick,green, name path=DG] (1,7) -- (7,4); %DG

    %nombrando las intersecciones
    \path[name intersections={of=DG and AH, by={P}}];
    \path[name intersections={of=DG and CF, by={Q}}];
    \path[name intersections={of=AH and BE, by={R}}];
    \path[name intersections={of=BE and CF, by={S}}];

    %coloreando de azul el cuadrado chico
    \draw[fill=cyan, ultra thick, opacity=0.3] (P) -- (Q) -- (S) -- (R) --cycle;

    %coordenadas de los otros cuadrados
    \coordinate (I) at (4.6,8.2);
    \coordinate (K) at (9.4,5.8);
    \coordinate (L) at (8.2,3.4);
    \coordinate (J) at (3.4,-0.2);

    %coloreando de azul los demas cuadrados
    \draw[fill=cyan, thick, opacity=0.3] (I) -- (C) -- (Q) -- (P) --cycle;
    \draw[fill=cyan, thick, opacity=0.3] (C) -- (K) -- (L) -- (Q) --cycle;
    \draw[fill=cyan, thick, opacity=0.3] (Q) -- (L) -- (B) -- (S) --cycle;
    \draw[fill=cyan, thick, opacity=0.3] (R) -- (S) -- (J) -- (A) --cycle;

    %muestra de construccion de los cuadrados
    \draw[red,thin] (A) -- (R) -- (E) --cycle;
    \draw[blue,thick] (E) -- (R) -- (P) -- (D) --cycle;

\end{tikzpicture}
    \captionof{figure}{Solución}
\end{minipage}
\par

La forma de solucionarlo fue viendo que se formaban cuatro triangulos y cuatro cuadrilateros como los resaltados en la Figura 1.2, y uniendo cada triangulo con un cuadrilatero se formaban 4 cuadrados más, obteniendo asi que el area del cuadrado azul es una quinta parte del cuadrado A,B,C,D, es decir $7.2 u^{2}$
\\
\textbf{1.5 Logicamente aclarado (p.18-19)}
\begin{excercise}
1.16 Se corrieron los 100 metros planos en los juegos olímpicos. Participaron en la final sólo cinco competidores: Bernardo, Diego, Ernesto, Antonio y Carlos. Fíjese si, partiendo de los siguientes datos, puede encontrar el orden en el que llegaron a la meta:
\\ A) Antonio no fue ni el primero ni el último.
\\ B) Antonio, sin embargo, quedó por delante de Bernardo.
\\ C) Carlos corrió más rápido que Diego.
\\ D) Ernesto fue más rápido que Antonio pero más lento que Diego.
\end{excercise}

\textbf{Solución}
\begin{center}
    \begin{tabular}{c|c|c|c|c|c}
        Corredor & 1 & 2 & 3 & 4 & 5\\ \hline
        Antonio &0&0&0&1&0 \\ \hline
        Bernardo &0&0&0&0&1 \\ \hline
        Carlos &1&0&0&0& \\ \hline
        Diego &0&1&0&0& \\ \hline
        Ernesto &0&0&1&0&0 \\ \hline
    \end{tabular}
    \captionof{figure}{Los unos representan la posicion que ocupó cada participante de la carrera}
\end{center}

\subsection{Los desafíos propuestos por Xochitl.}
El Ajaw K’inich y Xochitl paseaban por uno de los jardines y Xochitl propone que, juntos descubran el rango de las personas que vayan encontrando a su paso y elAjaw K’inich acepta el reto y se presentaron las siguientes situaciones.

1. Estaban tres personas y ellos le preguntaron a A: «¿Eres ocelopilli o teyanca-qui?». A respondió, pero tan confusamente, que el Ajaw K’inich, al igual que Xochitl, no pudo enterarse de lo que decía. Entonces el Ajaw K’inich preguntó a B: «¿Qué ha dicho A?». Y B le respondió: «A ha dicho que es teyacanqui». Pero en ese instante el tercer hombre, C, dijo: «No creas a B, que está mintiendo». La pregunta es, ¿qué son B y C?. El Ajaw K’inich le explica a Xochitl, su razonamiento para dar con la verdad, a partir de lo que dice C,sin embargo, ella le explica que no era necesario la respuesta de C para saber la verdad sobre B. ¿Cómo razonaron estos personajes?
\\
\textbf{Solución.} A y C son ocelopilli, mientras que B es teyacanqui.\\

El Ajaw K'inich razonó viendo las dos posibilidades:
\begin{itemize}
    \item Si C no dice la verdad, entonces B es ocelopilli, de donde se sigue que A es teyacanqui pero esto no puede suceder porque ellos no dicen la verdad por lo cual tendriamos un teyacanqui que dice la verdad y un ocelopilli que miente.
    \item si C dice la verdad entonces B es teyacanqui, por lo cual A es ocelopilli. 
\end{itemize}

Mientras que la forma de razonar de Xochitl fue:
\begin{enumerate}
    \item Si B miente entonces A es ocelopilli y dijo la verdad
    \item Si B dice la verdad entonces A es teyacanqui pero en ese caso A estaría mintiendo lo cual nos llevaría a que es un ocelopilli mentiroso
\end{enumerate}

2. Unos pasos más adelante, se encuentran a tres personas y le preguntan a la persona A, «¿Cuántos ocepilli hay entre ustedes»; la persona A contesta pero el canto de un cenzontle, provocó que su respuesta fuera ininteligible. Entonces Xochitl le pregunta B «A ha dicho que hay sólo hay un ocepilli entre nosotros» Y por su parte C les dice, «No le crean a B que está mintiendo». Ahora, ¿qué son B y C.

\textbf{Solución.} B es teyacanqui y C ocelopilli.

\begin{enumerate}
    \item Si B dice la verdad, entonces A también la dijo, pero esto no es posible porque se supone que solo hay un ocelopilli, llevando a que A y B se contradigan
    \item Si C dice la verdad, no hay ninguna contradicción
    \item Si C miente, entonces B dice la verdad y nos lleva al primer caso
\end{enumerate}

3. Un poco después se encuentran a dos persona más y le preguntan a una de ellas sobre su rango y contesta: Uno al menos de nosotros es teyacanqui. ¿Qué es la persona que contesta? ¿Y la otra?\\

\textbf{Solución.} El que contesta es ocelopilli y el otro es teyacanqui
\begin{itemize}
    \item El que contesta miente, pero esto no es posible pues de ser así entonces
    \begin{enumerate}
        \item Los dos son teyacanquis, por lo cual dijo la verdad
        \item Ninguno es teyacanqui, pero supusimos que el que contesta miente
    \end{enumerate}
    \item El que contesta dice la verdad
\end{itemize}

4. Apenas habían terminado de explicarse el acertijo anterior, cuando pasan dos personas y una de ellas dice: O yo soy teyacanqui ó mi acompañante es ocelopilli. ¿Qué es la persona que contesta y qué es su acompañante?

\textbf{Solución.} Ambas personas son ocelopilli

5. De vuelta al templo mayor, se encuentran con A, B y C. A dice: Todos nosotros somos teyacanquis. B dice: Uno de nosotros, y sólo uno es un ocelopilli. ¿Qué son A, B y C?\\

\textbf{Solución.} A y C son teyacanqui, B es ocelopilli.
\begin{enumerate}
    \item A miente, porque si dijera la verdad habria una contradicción
    \item Si B estuviese mintiendo entonces se cumpliría el caso 1
\end{enumerate}

6. Xochitl al Ajaw K’inich, le plantea lo siguiente, si las respuestas del grupo anterior hubieran sido: A: Todos somos teyancaquis. B: Uno y sólo uno, es un teyancanqui. ¿Puede determinarse lo que es B? ¿Puede determinarse lo que es C?\\

\textbf{Solución.} No sería posible determinar que es C
\begin{enumerate}
    \item A miente, porque si dijera la verdad habria una contradicción
    \item Si B estuviese mintiendo entonces se cumpliría el caso 1
    \item Si B dice la verdad entonces ese unico teyacanqui es A
\end{enumerate}

7. Posteriormente, se encuentran a dos y una de ellas dice: Yo soy teyacanqui y mi compañero no lo es. ¿Qué son A y B? \\

\textbf{Solución.} Los dos son teyacanquis

8. Casi llegando a palacio, se encuentran a tres personas; A y B dicen lo siguiente: A: B es teyacanqui. B: A y C son del mismo tipo. ¿Qué es C?\\
\textbf{Solución.} C es teyacanqui
\begin{enumerate}
    \item A dice la verdad entonces B miente y, A y C son diferentes
    \item A miente entonces C también es un mentiroso
\end{enumerate}


1.18 Xochitl posee tres tepozpetlacalli (una espacie de cofre), uno de onix, uno de ámbar y otro de obsidiana y en uno de ellos coloca sus hermosos pendientes de jade. Ella ha colocado una inscripción a cada tepozpetlacalli, las cuales dicen: En el de Onix: Los pendientes están aquí. En el de ámbar: Los pendientes no están aquí. En el de Obsidiana: Los pendientes no están en el de Onix. Ella le dijo al Ajaw K’inich: De los tres enunciados a lo sumo uno era verdad. El Ajaw K’inich superó bien la prueba, ¿qué cofre Figura 1.14 eligió?

\textbf{Solución.} Tenemos tres casos y que sol un enunciado es verdad, entonces

\begin{enumerate}
    \item Supongamos que el de onix dice la verdad, pero entonces el cofre de ámbar también está en lo correcto y no puedo haber más de un enunciado correcto.
    \item Supongamos que el de ámbar es correcto, entonces está en el de onix o el de obsidiana, pero si está en alguno de ellos volvemos a tener dos enunciados correctos.
    \item Tomando que el de obsidiana dice la verdad, entonces los pendientes solo pueden estar en el de ámbar para que dos enunciados sean falsos y uno verdadero.
\end{enumerate}

1.19 En el segundo reto los cofres tenían las siguientes inscripciones. En el de Onix: Los pendientes no están en el de ámbar. En el de ámbar: Los pendientes no están aquí. En el de Obsidiana: Los pendientes están aquí. Ella le dijo al Ajaw K’inich: De los tres enunciados por lo menos uno es verdadero y por menos otro es falso. ¿En cuál de los cofres están los pendientes?

\textbf{Solución.} Nuevamente, tenemos tres casos, pero ahora podemos tener dos enunciados verdaderos pero necesariamente uno debe ser falso
\begin{enumerate}
    \item Supongamos que el de onix dice la verdad, entonces el de ámbar dice la verdad, ahora, si los pendientes están en el de obsidiana tendríamos que el cofre de onix es falso y eso nos llevaría a una contradicción, por lo cual, para este caso los pendientes están en el de onix.
    \item Suponiendo verdadero el de ámbar entonces los pendientes solo pueden estar en el de onix o el de obsidiana y regresamos al caso anterior.
    \item Si el enunciado del cofre de obsidiana es verdadero caemos en que los otros dos también dicen la verdad y caemos en una contradicción, pues al menos uno es falso.
\end{enumerate}

Así, concluimos que los pendientes están en el cofre de onix.