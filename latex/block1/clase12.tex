\section{Clase 12. Resolución de la sección 1.5 del problemario}
\textbf{11/03/2025}

\textbf{Actividad 1.7}
En un cuadrado se localizan los puntos medios de sus lados y se trazan segmentos de sus vértices a los puntos medios como se muestra en la figura
¿ Cuánto vale él área del cuadrado A, B,C, D entre el área
del cuadrado azul?

\textbf{1.5 Logicamente aclarado (p.18-19)}
\begin{excercise}
1.16 Se corrieron los 100 metros planos en los juegos olímpicos. Participaron en la final sólo cinco competidores: Bernardo, Diego, Ernesto, Antonio y Carlos. Fíjese si, partiendo de los siguientes datos, puede encontrar el orden en el que llegaron a la meta:
\\ A) Antonio no fue ni el primero ni el último.
\\ B) Antonio, sin embargo, quedó por delante de Bernardo.
\\ C) Carlos corrió más rápido que Diego.
\\ D) Ernesto fue más rápido que Antonio pero más lento que Diego.
\end{excercise}

\textbf{Actividad 1.9}
El Ajaw K’inich y Xochitl paseaban por uno de los jardines y Xochitl propone
que, juntos descubran el rango de las personas que vayan encontrando a su paso y elAjaw K’inich acepta el reto y se presentaron las siguientes situaciones.

1. Estaban tres personas y ellos le preguntaron a A: «¿Eres ocelopilli o teyanca-qui?». A respondió, pero tan confusamente, que el Ajaw K’inich, al igual queXochitl, no pudo enterarse de lo que decía. Entonces el Ajaw K’inich preguntó a B: «¿Qué ha dicho A?». Y B le respondió: «A ha dicho que es teyacanqui». Pero en ese instante el tercer hombre, C, dijo: «No creas a B, que está mintiendo». La pregunta es, ¿qué son B y C?. El Ajaw K’inich le explica a Xochitl, su razonamiento para dar con la verdad, a partir de lo que dice C,sin embargo, ella le explica que no era necesario la respuesta de C para saber la verdad sobre B. ¿Cómo razonaron estos personajes?

2. Unos pasos más adelante, se encuentran a tres personas y le preguntan a la persona A, «¿Cuántos ocepilli hay entre ustedes»; la persona A contesta pero el canto de un cenzontle, provocó que su respuesta fuera ininteligible. Entonces Xochitl le pregunta B «A ha dicho que hay sólo hay un ocepilli entre nosotros» Y por su parte C les dice, «No le crean a B que está mintiendo». Ahora, ¿qué son B y C.

3. Un poco después se encuentran a dos persona más y le preguntan a una de ellas sobre su rango y contesta: Uno al menos de nosotros es teyacanqui. ¿Qué es la persona que contesta? ¿Y la otra?

4. Apenas habían terminado de explicarse el acertijo anterior, cuando pasan dos personas y una de ellas dice: O yo soy teyacanqui ó mi acompañante es ocelopilli. ¿Qué es la persona que contesta y qué es su acompañante?

5. De vuelta al templo mayor, se encuentran con A, B y C. A dice: Todos nosotros somos teyacanquis. B dice: Uno de nosotros, y sólo uno es un ocelopilli. ¿Qué son A, B y C?

6. Xochitl al Ajaw K’inich, le plantea lo siguiente, si las respuestas del grupoIntuición lógica y pensamiento lateral20 anterior hubieran sido: A: Todos somos teyancaquis. B: Uno y sólo uno, es un teyancanqui. ¿Puede determinarse lo que es B? ¿Puede determinarse lo que es C?

7. Posteriormente, se encuentran a dos y una de ellas dice: Yo soy teyacanqui y mi compañero no lo es. ¿Qué son A y B?

8. Casi llegando a palacio, se encuentran a tres personas; A y B dicen lo siguiente: A: B es teyacanqui. B: A y C son del mismo tipo. ¿Qué es C?

9. La pilli Xochitl, haciendo gala de sus habilidades, le plantea lo siguiente al Ajaw K’inich; la persona A dice que B y C son del mismo tipo. Al preguntarle a C «Son A y B del mismo tipo». ¿Qué responde C?