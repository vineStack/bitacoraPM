\section{Clase 1. Encuentra el algoritmo}
\textbf{11/02/2025}
\section{Inicio de clase}

Al comienzo de la clase la profesora nos pidió escribir un número de 7 cifras, luego hacer una serie de operaciones con sus cifras, y después de dictarle 6 de las 7 cifras, las cuales no fueron tachadas, ella adivinaría el número que cada uno tachó acorde a las cifras que dictaron. Algunos ejemplos que se dieron en clase fueron los siguientes:
    \\
    \begin{center}
        \begin{tabular}{c|c}
            cifras que no tachó & cifra tachadas\\
            890029 & 8\\
            312421 & 5\\
            677504 & 7\\
            695979 & 9\\
            410754 & 6\\
        \end{tabular}
    \end{center}
    
Luego, algunas de las preguntas/pistas que se plantearon para hallar el algoritmo que usaba la profesora fueron:\\

    \begin{itemize}
        \item ¿Cuáles son las restricciones?
        \item ¿Importa el orden de las operaciones con cada dígito?
        \item Cuáles son los primeros pasos del algoritmo?
        \item El algoritmo está en base 10
    \end{itemize}

\section{El algoritmo}
\subsection{Lenguaje cotidiano}
    \begin{enumerate}
        \item Escribe un número de n cifras que no comience ni termine con el dígito 0
        \item Escribe otro número, utilizando únicamente las cifras del número anterior, que cumpla con las mismas condiciones
        \item Resta el número menor al mayor. En el resultado, tacha un dígito distinto de 0
        \item Dicta las cifras que no fueron tachadas, sin importar el orden en el que se presenten
    \end{enumerate}
    Para encontrar el dígito tachado
    \begin{enumerate}
        \item Suma las cifras que no fueron tachadas
        \item Divide el resultado de la suma entre 9
        \item Si la división no es exacta, réstale al divisor el residuo de la división
        \item El resultado de esta resta es el dígito  que fue tachado
    \end{enumerate}

\subsection{Lenguaje matemático}
    \begin{enumerate}
        \item Escribe un número $A=a_1a_2...a_n$ de $n$ cifras con $a_1 \neq 0$ y $a_n\neq 0$
        \item Escribe un número $B=b_1b_2...b_n$ de $n$ cifras con $b_1 \neq 0$ y $b_n\neq 0$ el cual sea una permutación de las cifras de $A$
        \item Calcula $C=\|{A-B}$ y elige $x\in C$ tal que $x\neq 0$ y se tacha $x$
        \item Dicta el conjunto $C\ {x}$
    \end{enumerate}
    Para encontrar el dígito tachado
    \begin{enumerate}
        \item $S=\Sigma $ (dígitos C\ excepto $x$)
        \item $S/9 = q+r$
        \item $9-r = x$
    \end{enumerate}