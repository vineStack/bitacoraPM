\section{Clase 19}
\textbf{27/03/2025}

1.24 Una señora fue a comprar USBs de la marca Ynox. Al llegar a la tienda, pidió cierta cantidad y entregó el dinero correspondiente. La persona que la atendió le dijo: “Tenemos en promoción la marca Datafiel, que es mexicana. Con el dinero que me entregó, le doy la misma cantidad que pidió y tres más, todas de la marca Datafiel.” “Si me llevara las que le pedí y tres más, pero todas de la marca Ynox, ¿cuánto me faltaría?” “La quinta parte del dinero que ya me entregó.” Así que la señora se llevó las de la marca Datafiel. ¿Cuántas USBs compró la señora? ¿Se puede determinar en qué porcentaje es más barata la marca Datafiel que la marca Ynox?

1.25 “Así que te acabas de comprar zapatos, corbata y camisa” – dijo Eva a su esposo–. “Seguro que te has gastado bastante dinero.” Adán hizo un gesto de desaprobación. “Tan solo me he gastado mil pesos en total,” le explicó. “Los zapatos han costado el triple que la camisa y la corbata juntos”, ¿Cuánto costaron los zapatos?

1.26 Demuestra que: γ = α + β Usando propiedades de la función tangente, te será muy fácil demostrarlo. Te proponemos que hagas una construcción en la que visualmente se vea la igualdad pedida.

1.27 El IPN organizará los Interpolitécnicos de Boxeo en dos categorías: Nivel Medio Superior y Nivel Superior, cada una con rama femenil y varonil. Se espera la participación de 150 boxeadoras de Nivel Medio Superior y 180 boxeadoras de Nivel Superior. Asimismo, se anticipa la participación de 200 boxeadores de Nivel Medio Superior y 321 boxeadores de Nivel Superior. El torneo será por eliminación directa. En caso de que haya un número impar de participantes, se elegirá uno al azar para que pase a la siguiente ronda sin pelear. Se estima que cada pelea costará \$500 en concepto de arbitraje. ¿Cuánto se gastará en arbitraje para conocer a las dos campeonas y los dos campeones de boxeo? Para calcular el costo total de arbitraje, ¿se puede generalizar el resultado obtenido?.