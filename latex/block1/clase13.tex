\section{Clase 13.}
\textbf{11/03/2025}

\subsection{Los desafíos propuestos por Xochitl (Problemario p.20)}

1.18 Xochitl posee tres tepozpetlacalli (una espacie de cofre), uno de onix, uno de ámbar y otro de obsidiana y en uno de ellos coloca sus hermosos pendientes de jade. Ella ha colocado una inscripción a cada tepozpetlacalli, las cuales dicen: En el de Onix: Los pendientes están aquí. En el de ámbar: Los pendientes no están aquí. En el de Obsidiana: Los pendientes no están en el de Onix. Ella le dijo al Ajaw K’inich: De los tres enunciados a lo sumo uno era verdad. El Ajaw K’inich superó bien la prueba, ¿qué cofre Figura 1.14 eligió?

1.19 En el segundo reto los cofres tenían las siguientes inscripciones. En el de Onix: Los pendientes no están en el de ámbar. En el de ámbar: Los pendientes no están aquí. En el de Obsidiana: Los pendientes están aquí. Ella le dijo al Ajaw K’inich: De los tres enunciados por lo menos uno es verdadero y por menos otro es falso. ¿En cuál de los cofres están los pendientes?

\subsection{Problemas a resolver (Problemario p.21)}
1.1 Una orquesta de 20 músicos decide formar dos grupos musicales, uno de clásica y otro de música de salón. El primer grupo lo integran 8 personas y el segundo 12 personas. Si tres de los músicos pertenecen a los dos grupos ¿cuántos miembros de la orquesta original decidieron no pertenecer a ningún grupo?

1.2 Anote en el dibujo los valores de la cartas y palos de las tres cartas si se sabe que: Las tres cartas son un rey, un caballo y una sota. Los palos son oros, copas y espadas, aunque no necesariamente en el orden en que lo estamos diciendo. Oros está entre copas y el caballo. La sota está inmediatamente a la derecha de Figura 1.15: Descubre las las espadas. cartas de naipes

1.3 Edith y Judith son dos hermanas gemelas. Una de las dos, nos se sabe cual,siempre miente; la otra siempre dice la verdad. Me acerco a una de ellas y le pregunto: –¿Judith es la que miente? – –Sí – me responde. ¿Con quién de ellas hablé?

1.4 Un día en que el ascensor no funcionaba, me vi obligado a bajar por la escalera. Ya había descendido 7 escalones cuando vi al profesor Zivolotsk en la planta baja, aprestándose a subir. Sin discontinuar el descenso con mi habitual paso regular, saludé a Zivolotsk al cruzarme con él y, finalmente, comprobé con asombro que cuando aún me faltaban 4 escalones por bajar, el profesor acababa de alcanzar el rellano desde donde yo había empezado el descenso. «Parece el gato con botas » pensé; «cuando yo bajo un escalón, él sube dos de una zancada». ¿Cuántos escalones tiene la escalera? ¿en qué escalón nos cruzamos?

\begin{center}
    \textbf{Problemas de configuraciones geométricas}
\end{center}

\begin{minipage}[t]{0.48\textwidth}
    \noindent
    \begin{excercise}
        Acomoda 12 puntos en 6 filas y cada fila con 4 puntos.
    \end{excercise}
\end{minipage}
\begin{minipage}[t]{0.48\textwidth}
    \noindent
    \begin{center}
        % \input{../graficosTikz/b1Tikz/tickzC13/figure.tex}
    \end{center}
\end{minipage}
\par

\bigskip\hrule\bigskip

\par
\begin{minipage}[t]{0.48\textwidth}
    \noindent
    \begin{excercise}
        Acomoda 3 puntos en 12 filas, cada fila con 3 puntos
    \end{excercise}
\end{minipage}
\begin{minipage}[t]{0.48\textwidth}
    \noindent
    \begin{center}
        % \input{../graficosTikz/b1Tikz/tickzC13/figure.tex}
    \end{center}
\end{minipage}
\par

\bigskip\hrule\bigskip

\par
\begin{minipage}[t]{0.48\textwidth}
    \noindent
    \begin{excercise}
        Acomoda 16 puntos en 10 filas, cada fila con cuatro puntos.
    \end{excercise}
\end{minipage}
\begin{minipage}[t]{0.48\textwidth}
    \noindent
    \begin{center}
        % \input{../graficosTikz/b1Tikz/tickzC13/figure.tex}
    \end{center}
\end{minipage}
\par

\bigskip\hrule\bigskip

\par
\begin{minipage}[t]{0.48\textwidth}
    \noindent
    \begin{excercise}
        Acomoda 25 puntos en 12 filas, cada fila con 5 puntos.
    \end{excercise}
\end{minipage}
\begin{minipage}[t]{0.48\textwidth}
    \noindent
    \begin{center}
        % \input{../graficosTikz/b1Tikz/tickzC13/figure.tex}
    \end{center}
\end{minipage}