\section{Clase 15}
\textbf{11/03/2025}

1.10 El emperador de China, Yang Suen, planteó el siguiente desafío a sus ministros, prometiendo un premio al primero que lo resolviera: Una banda de 10 ladrones, compuesta por maestros y aprendices, robó 56 perlas. Cada maestro tomó 6 perlas y cada aprendiz tomó 5. Si no sobraron ni faltaron perlas, ¿cuántos maestros había en la banda? Recuerde que en esa época no existía el álgebra, por lo que se debe utilizar un razonamiento aritmético para encontrar la respuesta.

1.11 Dos coches salieron desde dos puntos distintos y en sentidos contrarios a correr una carrera de regularidad sobre un circuito cerrado. Cada coche mantuvo siempre fija su velocidad. Los coches se cruzaron por primera vez en cierto punto A. El segundo cruce se produjo en un punto B. El tercer cruce ocurrió en un punto C. Y el cuarto cruce se dio otra vez en el punto A. ¿Cuánto más rápido va un coche que el otro?

1.12 En una escuela japonesa, cada mañana, al inicio de clase, cada alumno debe hacer una reverencia como saludo a cada uno de sus colegas y a su profesor. Una mañana hubo un total de 625 saludos. ¿Cuántos alumnos había en el salón de clases?

1.13 Durante su entrenamiento, Isadora ha corrido durante 5 sábados seguidos. Cada sábado corre un kilómetro más que el anterior. Si en total ha corrido 60 km, ¿cuántos kilómetros corrió el quinto sábado?

1.14 Se tienen dos soluciones de agua oxigenada: una al 30 \% y otra al 3 \%. Se deben mezclar de tal manera que se obtenga una solución al 12 \%. ¿Cómo deben mezclarse estas soluciones de agua oxigenada?

1.15 Un barco se desplaza 5 horas sin interrupción río abajo desde la ciudad A a la ciudad B. De vuelta avanza contra la corriente (con su marcha ordinaria y sin detenerse) durante 7 horas. ¿Cuántas horas necesitará una balsa para desplazarse de la ciudad A a la B, yendo a la misma velocidad de la corriente?