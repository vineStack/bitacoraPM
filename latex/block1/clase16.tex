\section{Clase 16}
\textbf{19/03/2025}

1.6 Todos los miércoles la señora Minouflet reúne a sus nietos al- rededor de una merienda; mujer fiel a sus costumbres, para esta ocasión compra un número invariable de pastelitos que distribuye siempre de manera igual entre los niños. Un miércoles, Arsène no acude a la reunión y cada chico recibe dos pastelitos de más. El miércoles siguiente, Arsène llega con un compañero y cada niño recibe un paste- lito de menos. ¿Cuántos nietos tiene la señora Minouflet?

1.17 Tres cajas contienen: una, dos bolas blancas; otra, dos bolas negras; y la tercera, una bola blanca y una bola negra. Las etiquetas BB, NN y BN han sido equivocadamente pegadas, de modo que ninguna de las cajas lleva la etiqueta correcta. Para restituir a cada caja la etiqueta que le corresponde, se permite sacar solamente una bola de una caja. ¿Cuál caja debe elegirse?

1.18 Una vieja leyenda árabe cuenta que un día Mustafá se encontró con un genio,quien le propuso duplicar la cantidad de monedas de oro que poseía a cambio de quedarse con 200 monedas. Mustafá aceptó con gusto. El genio le propuso hacer lo mismo nuevamente y Mustafá aceptó. Una vez más, el genio insistió en hacer lo mismo y Mustafá aceptó, pero cuando revisó el cofre de monedas, se percató de que no quedaba ninguna. ¿Cuántas monedas tenía Mustafá al encontrarse con el genio?

1.19 Clovis Clou a veces toma el subterráneo en Châtelet. Utiliza el pasillo rodante, sobre el cual camina a su paso habitual, y de esta manera recorre de un extremo a otro en 1 minuto y 12 segundos. Un día, al regresar, tuvo la curiosa idea de caminar hacia atrás en esa misma cinta transportadora, manteniendo siempre su mismo paso. Para recorrer esa distancia necesitó seis minutos. Al día siguiente, el pasillo rodante no funcionaba debido a un desperfecto. ¿Cuánto tiempo empleó Clovis entonces para cubrir esa distancia?