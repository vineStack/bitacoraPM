\section{Clase 18}
\textbf{26/03/2025}

1.20 Un vinicultor, reúne a sus cuatro hijos y les pide que dividan el dominio heredado en cuatro partes iguales; El dominio en cuestión era un trapecio rectangular. La base menor medía lo mismo que la altura del trapecio y media la mitad de la base mayor. Indicar como se debe proceder para que se cumpla la voluntad del vinicultor.

1.21 En las obras de un matemático árabe del siglo XI hallamos el siguiente problema: A ambas orillas de un río crecen dos palmeras, la una frente a la otra. La altura de una es de 30 codos, y la de la otra, de 20. La distancia entre sus troncos, 50 codos. En la copa de cada palmera hay un pájaro. De súbito los dos pájaros descubren un pez que aparece en la superficie del agua, entre las dos palmeras. Los pájaros se lanzaron y alcanzaron el pez al mismo tiempo. ¿A qué distancia del tronco de la palmera mayor apareció el pez?

1.22 Tenemos la siguiente figura con forma de L,hay que dividir la figura en:Intuición lógica y pensamiento lateral 24
1. Dos partes de igual tamaño.
2. Tres partes de igual tamaño.
3. Cuatro partes de igual tamaño.

1.23 En la oficina de Silvia hay dos impresoras láser: la Rapidix y la Esaeslentix. Ella necesita imprimir un documento con una gran cantidad de hojas y, al hacer sus cálculos, estima que usar la impresora Esaeslentix tomaría 7 minutos más que usar la impresora Rapidix. Sin embargo, Silvia distribuyó el número de hojas del documento de manera proporcional a las velocidades de las impresoras y puso a trabajar ambas. De esta forma, todo el documento estuvo impreso en 12 minutos. ¿Cuánto tiempo se tardaría en imprimir todo el documento si solo se utiliza la impresora Esaeslentix?